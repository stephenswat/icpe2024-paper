\section{Reproducibility and Reusability}

The evolutionary algorithms, scripts for the processing and visualisation of data, and other software used in this paper are permanently archived on Zenodo~\cite{swatman_2024_10567243}, and have been made available at \doi{10.5281/zenodo.10567243}. The aforementioned artifact also contains all data that was gathered and processed during the work presented in this paper. For more information about the use of the artifact accompanying this paper, see the included \texttt{README} file.

\section{Conclusions and Future Work}

\label{sec:conclusion}

In this paper, we have discussed a generalization of the Morton layout for multi-dimensional data and we have shown that there exist families of array layouts with strongly varying cache behavior which, in turn, impact the performance of applications. We have shown how these layouts can be systematically described, and that the number of possible layouts quickly exceed the limits of what can be feasibly explored using exhaustive search. We have proposed a method based on evolutionary algorithms for the exploration of the design space of such layouts. We have evaluated the fitness of different array layouts using cache simulation and we have presented results indicating that our fitness function correlates with real world performance. Furthermore, we have shown that the methodology described in this paper can be used to improve the performance of applications on real hardware by up to ten times.

In the future, we intend to investigate the use of multi-objective optimization using NSGA-II~\cite{996017} in order to find array layouts that provide favorable cache behavior across multiple applications. We also intend to explore more advanced genetic algorithms which are known to perform well in combinatorial problems, such as RKGA~\cite{doi:10.1287/ijoc.6.2.154} and BRKGA~\cite{Goncalves2011}. It is our belief that exploring more evolutionary strategies will give us more insight into the convergence properties of various methods, and allow us to select the most efficient one. Although our fitness function correlates with real-world performance, the correlation is not perfect; we believe that the efficacy of our method could be improved through the development of more advanced fitness function, perhaps through the use of machine learning methods. In particular, we believe that the field of metric learning may enable us to develop more accurate fitness functions, and we aim to explore this avenue of research in the future. Finally, we aim to expand our research to a broader range of access patterns and hardware, including graphics processing units (GPUs).
